\documentclass[a4wide,12pt]{report}

\usepackage[romanian]{babel}
\usepackage[utf8x]{inputenc}
\usepackage{setspace}
\usepackage{graphicx}
\usepackage[top=3cm, bottom=3cm, left=3cm, right=3cm]{geometry}
\usepackage{tikz}
% Use Open Sans as the sans serif font.
\usepackage[defaultsans,osfigures,scale=0.95]{opensans}
\usepackage[T1]{fontenc}
\usepackage{color}
\usepackage{relsize}        % Pentru mărimi relative de text.
\usepackage{pgfplots}

\renewcommand{\seriesdefault}{l}

% Opțiuni la diagrame tikz.
\usetikzlibrary{patterns, trees, matrix, arrows, decorations.pathmorphing,
decorations.pathreplacing, plotmarks}

% Comenzi
\newcommand{\eng}[1]{\emph{#1}} % Pentru cuvinte în limba engleză.
\newcommand{\cod}[1]{\texttt{#1}}
\newcommand{\url}[1]{\texttt{#1}}
\newcommand{\uplu}[1]{$\langle$#1$\rangle$}
\newcommand{\floor}[1]{\lfloor{}#1\rfloor} % Pentru trunchere.
\newcommand{\acr}[1]{{\textsmaller[1]{\textsc{#1}}}} % Pentru acronime.
\newcommand{\idee}[1]{{\color{red} #1}}
\newcommand{\poate}[1]{#1}

\newcommand{\capfara}[1]{
    \chapter*{#1}
    \addcontentsline{toc}{chapter}{#1}
}

\newcommand{\secfara}[1]{
    \section*{#1}
    \addcontentsline{toc}{section}{#1}
}

% Variabile
\newcommand{\subtitlu}{editor web colaborativ de \LaTeX}
\newcommand{\titlu}{Multilatex: \subtitlu}
\newcommand{\autor}{Paul Răzvan Nechifor} % Ăsta-s io. :)
\newcommand{\coordonator}{conf. dr. Sabin-Corneliu Buraga}
\newcommand{\sesiunea}{iulie, 2014}
\newcommand{\datacurenta}{\today}
\newcommand{\loculsisemnatura}{
    \vspace{20mm}
    \noindent
    Iași, \datacurenta
    \vspace{20mm}
    \begin{flushright}
        Absolvent \emph{\autor}\\
        \setlength{\unitlength}{1cm}
        \begin{picture}
            (4, 1)\put(0, 0)
            {\line(1, 0){4}}
        \end{picture}
        \vspace{5mm}\\
        (semnătura în original)
    \end{flushright}
}

\begin{filecontents*}{statcounter.csv}
in out
2009.0  0.67
2009.5  1.05
2010.0  1.56
2010.5  2.86
2011.0  4.30
2011.5  7.02
2012.0  8.49
2012.5  11.09
2013.0  14.13
2013.5  17.35
2014.0  23.00
\end{filecontents*}

\begin{document}
\frenchspacing % Don't use double spacing.
\onehalfspace % Larger line space.

\fontdimen2\font=5.00pt % Interword space (default 3.91663pt).
\fontdimen3\font=2.00pt % Interword stretch (default 1.95831pt).
\fontdimen4\font=1.50pt % Interword shrink (default 1.30554pt).
\fontdimen7\font=0.00pt % Extra space (default 1.30554pt).

\pagestyle{empty} % Hide page number.

% Cover background color.
\begin{tikzpicture}[overlay]
    \definecolor{albastru de fii}{rgb}{0.176, 0.447, 0.713}
    \fill [fill=albastru de fii] (-5,5) rectangle (21.0,-29.7);
\end{tikzpicture}

\begin{figure}[!htb]
\minipage{2cm}
    \includegraphics[width=2cm]{imagini/uaic}
\endminipage\hfill
\minipage{11cm}
    \centering
    \color{white}
    \large{\textsf{Universitatea Alexandru Ioan Cuza Iași}}

    \LARGE{\textsf{Facultatea de Informatică}}

    \large{\textsf{Master Ingineria Sistemelor Soft}}
\endminipage\hfill
\minipage{2cm}%
    % FCS logo.
    \begin{tikzpicture}[scale=0.26]
        \def\taie{0.5}
        \definecolor{alb}{rgb}{1, 1, 1}

        \fill [fill=alb]
            (0,0) -- (2,0) -- (2,3) -- (3,3) -- (5,5) -- (2,5) --
            (2,6) -- (6,6) -- (8,8) -- (0,8) -- cycle;
        \fill [fill=alb]
            (3,0) -- (3,3-\taie) -- (5,5-\taie) -- (5,0) -- cycle;
        \fill [fill=alb]
            (6,0) -- (6,6-\taie) -- (8,8-\taie) -- (8,0) -- cycle;
    \end{tikzpicture}
\endminipage
\end{figure}

\vspace{2cm}

\begin{center}
    \centering
    \color{white}
    {\fontsize{22}{26}\selectfont \textsf{LUCRARE DE DIZERTAȚIE}}\\
    \vspace{1.5cm}
    {\fontsize{40}{46}\selectfont \textsf{\textbf{Multilatex}}}\\
    \vspace{0.5cm}
    {\fontsize{26}{28}\selectfont \textsf{\subtitlu}}\\
    \vspace{1.5cm}
    \large{\textsf{propusă de}}\\
    \vspace{1.5cm}
    {\fontsize{26}{28}\selectfont \textsf{Paul Răzvan \textbf{Nechifor}}}\\
    \vspace{1.5cm}
    {\fontsize{16}{26}\selectfont \textsf{\textbf{Sesiunea:} \sesiunea}}\\
    \vspace{1.5cm}
    {\fontsize{14}{26}\selectfont \textsf{Coordonator științific}}\\
    {\fontsize{18}{26}\selectfont \textsf{\textbf{\coordonator}}}\\
\end{center}

\pagebreak

% Pagina de titlu
\begin{center}
    \textbf{UNIVERSITATEA ALEXANDRU IOAN CUZA IAȘI}\\
    \vspace{3mm}
    \textbf{FACULTATEA DE INFORMATICĂ}\\
    \vspace{89mm}
    \Huge{\textbf{\titlu}}\\
    \vspace{30mm}
    \Large{\textbf{\emph{\autor}}}\\
    \vspace{11mm}
    \Large{\textbf{Sesiunea:} \sesiunea}\\
    \vspace{10mm}
    \normalsize{Coordonator științific}\\
    \vspace{3mm}
    \large{\textbf{\coordonator}}\\
    \end{center}
\pagebreak

\section*{Declarație privind originalitate și respectarea drepturilor de autor}

Prin prezenta declar că lucrarea de licență cu titlul „\titlu“ este scrisă de
mine și nu a mai fost prezentată niciodată la o altă facultate sau instituție de
învățământ superior din țară sau străinătate. De asemenea, declar că toate
sursele utilizate, inclusiv cele preluate de pe internet, sunt indicate în
lucrare, cu respectarea regulilor de evitare a plagiatului:

\begin{itemize}

\item toate fragmentele de text reproduse exact, chiar și în traducere proprie
din altă limbă, sunt scrise între ghilimele și dețin referința precisă a sursei;

\item reformularea în cuvinte proprii a textelor scrise de către alți autori
deține referința precisă;

\item codul sursă, imagini etc. preluate din proiecte cu sursă publică sau alte
surse sunt utilizate cu respectarea drepturilor de autor și dețin referințe
precise;

\item rezumarea ideilor altor autori precizează referința precisă la textul
original.

\end{itemize}

\loculsisemnatura

\pagebreak

\section*{Declarație de consimțământ}
Prin prezenta declar că sunt de acord ca lucrarea de licență cu titlul „\titlu“,
codul sursă al programelor și celelalte conținuturi (grafice, multimedia, date
de test etc.) care însoțesc această lucrare să fie utilizate în cadrul
Facultății de Informatică.

De asemenea, sunt de acord ca Facultatea de Informatică de la Universitatea
Alexandru Ioan Cuza Iași să utilizeze, modifice, reproducă și să distribuie în
scopuri necomerciale programele-calculator, format executabil și sursă,
realizate de mine în cadrul prezentei lucrări de licență.

\loculsisemnatura

\pagebreak

% Cuprins
\pagestyle{plain}
\setcounter{page}{1} % Începe numerotarea de la 1.

\tableofcontents

\pagebreak

\capfara{Introducere}

\secfara{Motivație}

\secfara{Challanges}

\capfara{Contribuții}

\chapter{Sisteme existente}

\section{\TeX}

\TeX{} este un program de compoziție tipografică care a fost început în 1977 de
către Donald Knuth pentru a permite o compoziție mai bună a articolelor și
cărților științifice.

Codul sursă pentru \TeX{}\cite{texweb} este scris în WEB, un limbaj de
programare literară\footnote{Un tip de programare prin care codul sursă este
îmbinat descrierea lui astfel încât să poată fi citit similar cu o carte.}
similar cu Pascal. Folosind acest tip de programare codul sursă poate fi
compilat într-un fișier PDF\cite{texwebpdf}.

\section{\LaTeX}

\LaTeX{} este un set generalizat de macrouri bazat pe \TeX{}.

Spre diferența de editoarele WYSIWYG, \LaTeX{} are avantajul că permite
concentrarea pe conținut și nu pe presentarea lui. În editoare WYSIWYG
utilizatorii sunt temptați să folosească spații și linii goale pentru aliniere
și \cod{page breaking} pentru că editează prezentarea în loc de semantică.

Deși era un program bun la conceperea lui pe sistemele limitate de atunci,
astăzi \LaTeX{} nu exploatează avantajele aduse de sistemele moderne.

Unele dintre problemele pe care urmează să le prezint sunt rezolvate de situri
precum ShareLatex și WriteLatex. În capitolul următor am să descriu sistemul
propriu.

\subsection{Modul de lucru}

\LaTeX{} are un mod de lucru static: editarea fișierelor, compilarea,
vizualizarea documentului compilat, repetarea. \LaTeX{} nu suportă doar
recompilarea modificărilor ci necesită recompilarea întregului document ceea ce
este încet. Mai rău este că este nevoie de mai multe compilări dacă se folosește
un cuprins, bibliografie, referințe și altele.

Această problemă nu poate fi rezolvată ușor deoarece ar necesita rescrierea
aplicațiilor de bază.

\subsection{Memorarea stării}

\LaTeX{} operează prin folosirea și modificarea unor fișiere în dosarul de
lucru. Acest lucru complică orice program care vrea să-l folosească pentru că 

\subsection{Sistemul de pachete}

\LaTeX{} are suport limitat pentru pachete. Nu există versionarea declararea
folosirii pachetelor, deci compilarea unui document poate duce la rezultatea
diferite (sau la erori) într-un mod neașteptat.

\idee{Repositories are based on distributions.}

\section{ShareLatex}

\section{WriteLatex}

\chapter{Sistemul propriu}

\section{Proiectare}

În scrierea sistemului propriu am început întâi cu proiectarea pe care am
descris-o pe situl proiectului la \cod{multilatex.com/blog} . Urmează
principalele probleme asupra cărora a trebuit să mă decid.

\subsection{Documente mari}

Deseori, un document \LaTeX{} mare este împărțit în mai multe fișiere \cod{.tex}
care sunt incluse în cel principal. Deși acest lucru este suportat, eu am ales
să nu încurajez acest lucru pentru că am observat că întră în conflict cu modul
meu de lucru: când încep un document nu specific perfect structura lui și ajung
des să mut secțiuni și paragrafe prin el.

Principala problemă la a folosi un fișier monolitic este greutatea de navigare,
deci am ales să rezolv problema asta prin a implementa în panoul arborelui de
fișiere un nivel suplimentar prin care fișierele \cod{.tex} listează toate
capitolele, secțiunile și subsecțiunile. Astfel se poate sări ușor printre
părțile documenului.

\subsection{Editarea colaborativă}

Există două modalități de editare colaborativă: în timp real (precum Google
Docs) sau bazată pe commit-uri (precum Git sau alte sisteme de control a
versiunilor).

Deoarece \LaTeX{} nu funcționează precum un editor WYSIWYG, a doua variantă pare
mai bună, dar operațiile de integrare ar complica mult sistemul.

Sunt problema și la colaborarea în timp real. Aici se complică operațiile de
anulare (\eng{undo}) deoarece nu există un istoric bine definit. În plus, nu se
poate decide ușor care sunt versiunile intermediare.

Am ales până la urmă să le împlementez pe ambele variante. Prima folosind
Share.JS și a doua folosind MongoDB și un \eng{file store} propriu. În
retrospectivă, acest lucru a complicat mult sistemul.

\subsection{Compararea modificărilor}

O altă problemă pe care mi-am pus-o la proiectarea a fost compararea versiunilor
intermediare și contribuțiilor de la fiecare persoană. Folosind doar
\eng{commit}-uri acest lucru este ușor, dar cu versiune în timp real este greu
să fie revizuite contribuțiile fiecărei persoane deoare ce se pot suprapune
foarte mult.

Ca soluție am decis ca să permit \eng{commit}-uri cu mai mulți autori. Astfel
când se face un \eng{commit}, fiecare persoană care a făcut o modificare între
timp va apărea ca un autor. Dezavantajul este că nu se poate stabili pentru o
versiune ce modificare a făcut fiecare dintre autori, deci trebuiesc făcute
\eng{commit}-uri dese.

\subsection{Compilatorul de \LaTeX}

Am considerat și folosirea unui compilator de \LaTeX{} care să genereze
PDF-urile pe partea de client. Spre exemplu \cod{texlive.js}\cite{texlivejs}
poate compila proiecte direct din navigator. Acest lucru este posibil pentru că
\TeX{} Live a fost compilat în JavaScript folosind Emscripten.

Problemele majore sunt că este prea încet și nu suportă toate pachetele \TeX{}
Live. Acest lucru chiar ar fi imposibil deoarece o instalare completă a
\cod{texlive-full} are mai mult de 2~GiB.

\subsection{Importare și exportare}

Un factor important pentru succesul unui astfel de sit este compatibilitatea și
încrederea în permanența conținutului, deci trebuie să existe un mecanism ușor
de a importa și de a exporta proiectele. Acest lucru este parțial complicat de
faptul că proiectele \LaTeX{} pot avea o multitudine de fișiere. Pentru a
rezolva această problemă am ales să consider că fișierul denumit \cod{main.txt}
sau singurul fișier \cod{.tex} este cel principal.

\subsection{Altele}

Alte probleme au fost legate de tehnologiile pe care să le folosesc pentru
construirea proiectului. Aceste programe/librării vor fi prezentate mai departe
în acest capitol.

\section{Structura sitului}

În ceea ce urmează am să descriu cum am structurat paginile, conținutul și
organizarea sitului.

Consider că audiența principală a sitului va fi construită din persoane care au
mai folosit \LaTeX{} în trecut și doresc să se mute în \eng{cloud} și persoane
care vor să colaboreze mai bine în editarea de proiecte \LaTeX{}.

\subsection{Prima pagină}

Rolul primei pagini este de a explica într-un mod foarte rapid ce se poate face
cu acest sit și să-i convingă pe potențialii utlizatori să se înscrie. Pentru
asta am făcut posibil o înscriere cât mai rapidă (vezi secțiunea~\ref{inregrap}
la pagina \pageref{inregrap}).

Principala funcționalitate este un \eng{jumbotron} cu o demonstrație și
formularul de înregistrare rapidă.

Alte lucruri de luat în considerare ar fi recomandandări de la alți utilizatori.

\subsection{Galerie}

Galeria prezintă cele mai populare proiecte. Într-un fel, rolul ei este de a
lista cele mai bune creații și de a demonstra ceea ce este posibil cu acest
proiect. Pagina este structurată ca o grilă de miniaturi.

Luând inspirație de la Spreaker Deck, am decis să folosesc și eu o
previzualizare dinamică a proiectelor. În loc să folosesc o pagină predefinită
ca miniatură pentru un document, la compilarea unui \eng{commit} fac miniaturi
la toate paginile și mișcarea pe orizontală asupra miniaturii decide care pagină
să fie afișată. Astfel se poate previzualiza ușor un document fără să fie
deschis.

Tot aici sunt listate și șabloanele precompilate (secțiunea~\ref{sabloanesec},
pagina~\pageref{sabloanesec}) de unde pot fi și duplicate ușor.

\subsection{Editor}

\subsection{Istoric}

\subsection{Pagina utilizator}

Nu am intenționat să fac un sit social deci pagina unui utilizator nu conține
decât metadate publice și listări a proiectelor și activității lui.

\subsection{\acr{URL}-uri}

Cred că proiectarea \acr{URL}-urilor este importantă pentru că ajută la
folosirea mai ușoară a sitului. Un exemplu de situri cu \acr{URL}-uri bune este
GitHub unde dacă numele meu de utilizator este \cod{paul-nechifor} și numele la
un proiect este \cod{multilatex} \acr{URL}-ul pentru el este
\cod{github.com/paul-nechifor/multilatex}. Ceea ce este ușor de memorat și ușor
de transmis în medii non-electronice. A se compara cu alte variante des
întâlnite precum un ipotetic
\cod{github.com/projects.php?user=51234\&project=654742}.

Urmează exemple a \acr{URL}-urilor folosite unde \cod{:nume} reprezintă o
variabilă.

\begin{description}

\item[\cod{/}] Pagina principală.

\item[\cod{/api/...}] Toate paginile de folosite de \acr{API}.

\item[\cod{/blog}] Cele mai recente articole din blog.

\item[\cod{/blog/:post}] Un articol specific.

\item[\cod{/explore}] Pagina de explorare a galeriei de proiecte.

\item[\cod{/:username}] Pagina personală a unui utilizator.

\item[\cod{/:username/:location}] Pagina unui proiect.

\item[\cod{/:username/:location/commit/:n}] Pagina unui proiect la o anumită
versiune (\eng{commit}).

\item[\cod{/:username/:location/commit/:n/fork}] \acr{URL}-ul de duplicare a
unui proiect la o anumită versiune.

\item[\cod{/:username/:location/head/pdf}] Vizualizarea documentului compilat
(varianta curentă din editor).

\item[\cod{/:username/:location/head/log}] Vizualizarea logului de compilare.

\item[\cod{/:username/:location/zip}] Descărcarea proiectului în format ZIP.

\end{description}

\subsection{Dispozitive mobile}

Datorită creșterii utilizării dispozitivelor mobile pe web
(figura~\ref{mobilefig}) este important să fie luate în considerare.

Pentru realizarea sitului am utilizat o proiectare web adaptabilă
(\eng{responsive}) folosind interogări \cod{@media}. Singura excepție este
pagina editorului deoarece necesită un ecran mare.

\idee{Imagine cu versiunea desktop vs mobile.}

\begin{figure}[hb]
\begin{center}
\begin{tikzpicture}
    \begin{axis}[
        width=0.9\textwidth,
        height=4cm,
        ymajorgrids,
        x tick label style={/pgf/number format/1000 sep=},
        ytick scale label code/.code={},
        max space between ticks=50pt,
    ]
        \addplot table[x=in,y=out]{statcounter.csv};
    \end{axis}
\end{tikzpicture}
\end{center}
\caption{Utilizarea dispozitivelor mobile pe Web (ca procent)\cite{statcount}.}
\label{mobilefig}
\end{figure}

\subsection{Stil}

Stilul sitului nu se numără printre cele mai importante aspecte și am ales să
folosesc Bootstrap 3 cu puține modificări pentru o estetică plată și simplă.

\section{Integrare continuă}

\idee{rsync}

\section{Construirea}

Construirea proiectului a susținut cele mai mari modificări și a trecut prin
trei variante. Prin construire mă refer la compilarea, configurearea și
instalarea părților necesare.

\subsection{Programul propriu}

Inițial am încept prin a scrie propriul program pentru a face construirea.
Programul folosea pachetul \cod{commander} pentru a procesa argumentele (spre
exemplu pentru a suprascrie opțiuni din fișierul de configurație) și avea
opțiunile de a instala și de a face \eng{deployment}.

Instalarea implica copierea fișierelor prin \cod{rsync}, crearea dosarelor
speciale și repornirea serviciului Upstart. Atunci nu făceam compilare de niciun
fel.

\eng{Deployment}-ul implica copierea dosarului de dezvoltare pe un sistem
specificat și rularea scriptului de instalare.

După asta am descoperit Grunt care rezolvă această problemă într-un mod general.

\subsection{Grunt}

Grunt este un sistem de automatizare a proceselor de lucru cu foarte multe
\eng{plugin}-uri care funcționeaza prin descrirea de sarcini de lucru.

Un avantaj peste folosirea de fișiere \cod{Makefile} este că Grunt este scris în
JavaScript și modulele sunt făcute să fie independente de platformă.

\subsubsection{Compilare Stylus}

Fișierele Stylus pot fi compilate la \eng{runtime} de Express, dar este mai
eficient să fie compilate dinainte în CSS.

\subsubsection{Unificare JavaScript}

În mod normal fișierele JavaScript sunt incluse individual in pagini prin
etichete \cod{script}. Acest lucru obligă trimiterea unei cereri pentru fiecare
fișier. În HTTP~1.1, nu mai este necesară creearea unei conexiuni noi per
cerere, dar tot există \eng{overhead} pentru cereri.

O altă problemă este lipsa de modularizare. Un modul logic compus din mai multe
fișiere JavaScript trebuie inclus în fiecare pagină. Astfel pot exista coliziuni
de variabile și includerea etichetelor \cod{script} în ordine greșită poate duce
la erori.

Pachetul Browserify permite compunerea fișierelor JavaScript într-unul singur
folosind specificația CommonJS de descriere a modulelor (cea folosită și de
Node). Astfel, în loc să fie descris un pachet în mod liniar prin etichete
\cod{script} în fiecare pagină în care este inclus, fiecare fișier descrie ce
dependințe are și Browserify compune tot codul necesar în funcție de fișierul de
intrare și dependințele acestuia în mod recursiv.

Un dezavantaj la acest mod de lucru este că se complică procesul de depanare
doarece nu mai corespund liniile la \eng{runtime} cu liniile din fișierele
proiectului. Din fericire acest lucru este rezolvat prin folosirea de mapări la
sursă (\eng{source maps}) unde pe lângă codul modulului unificat este inclusă în
comentarii și maparea la codul inițial. Însă recunoașterea mapărilor este
prezentă doar în navigatoarele recente.

\subsubsection{Minificare}

Pentru fișierele JavaScript și CSS se poate realiza o optimizare prin care se
scot toate spațiile și comentariile inutile. Operații mai avansate de optimizare
reprezintă redenumirea variabilelor locale la JavaScript sau eliminarea de cod
nefolosit (cod după \cod{return} în JavaScript sau opțiuni suprascrise de mai
multe ori la CSS).

\subsubsection{Instalarea și \eng{deployment}-ul}

De data aceasta pașii pentru instalare sunt descriși in JavaScript. După
compilarea fișierelor, ele sunt copiate în dosarul necesar (folosind \cod{rsync}
pentru evitarea copierii fișierelor neschimbate) și se execută restul pașilor
(crearea dosarelor și utilizatorilor dacă e nevoie, repornirea serviciului).
\eng{Deployment}-ul este similar variantei precedente.

\subsubsection{Altele}

Alte sarcini descrise sunt de compilare a șabloanelor de proiecte \LaTeX{} și
listarea fișierelor de logare în timp real de pe mașina țintă.

\subsubsection{Problema}

Problema la Grunt este că preferă configurația înainte de cod. Acest lucru duce
la configurații mari și organizate pe operații în loc de ordinea logică a
construirii.

\subsection{Gulp}

Gulp este un sistem de automatizare similar, dar cu diferența principală că
sarcinile sunt descrise prin fluxuri de transformare și se preferă cod înainte
de configurații stufoase.

Astfel construirea proiectului poate fi descrisă printr-un graf de dependințe și
transformări din care se execută doar cele necesare.

\idee{Un grafic cu transformările.}

\subsubsection{CoffeeScript}

A treia iterație a fost și momentul în care am început să folosesc CoffeeScript.
Însăși fișierul de configurare Gulp este scris în acest limbaj. Browserify poate
fi făcut să recunoască fișiere CoffeeScript și să le compileze înainte de
unificare.

\subsubsection{Bower}

În variantele precedente am inclus codul pentru modulele necesare (Ace, PDF.js
etc) direct în codul sursă a proiectului. Acest lucru duce la supradimensionarea
proiectului cu cod care nu-mi aparține. Pentru foarte mult timp acest mod de
lucru era normal pentru că nu exista un mod de a administra pachetelele
\eng{front-end}.

Bower este un astfel de manager de pachete pentru web.

\idee{TODO}

\section{\eng{Back-end}}

\subsection{Vagrant}

Imaginea sistemului de operare folosită este Ubuntu 12.04 deoarece era varianta
cu suport pe termen lung când am început proiectul.

\subsection{Upstart}

Programul construit rulează pe \eng{backend} ca un serviciu în Upstart. Upstart
este un sistem de administrare a serviciilor pentru Unix bazat pe evenimente. El
a fost proiectat ca un înlocuitor pentru init.

Am ales Upstart pentru că este instalat inițial în Ubuntu 12.04. Din următoarele
versiuni, însă, Ubuntu și Debian se vor schimba la Systemd care are mai multe
funcționalități.

Eu folosesc Upstart prin scriptul care specifică:

\begin{itemize}

\item ce utilizator să fie folosit pentru executarea programului (ca să nu fie
utilizat \cod{root} pentru sigurantă;

\item cum să repornească serviciul în caz de erori (se abandonează repornirea în
cazul în care eșuează mai multe de 10 ori în 5 secunde);

\item spre ce fișier de logare să fie redirectate ieșirile.

\end{itemize}

\subsection{Node}

\subsection{Express}

\subsection{Jade}

\subsection{Stylus}

Stylus (2010) este un limbaj \eng{stylesheet} care compilează în CSS. Este
asemănător cu Sass (2007) și LESS (2009), dar cu diferența principală că este
mult mai succint și este bazat pe indentare. Totuși, Stylus suportă și sintaxa
CSS.

În următorul exemplu este arătat codul Stylus în stânga și rezultatul CSS în
dreapta. Se folosește o variabliă \cod{main-color} pentru culoare care este și
manipulată pentru \cod{:hover} (înalbită cu 20~\%). Folosind \cod{main-size}
mărimea textului pe dispozitive cu ecran mic este micșorată la 80~\%. Este
demonstrată și imbricarea care evită repetarea: se folosește doar \cod{\&:hover}
în loc de \cod{a:hover, code:hover}.

\minipage{9cm}
\begin{verbatim}
main-color = #f7a
main-size = 18px

a, code
  color main-color
  font-size main-size
  @media (max-width: 479px)
    font-size floor(main-size * 0.8)
  &:hover
    color lighten(main-color, 20%)
\end{verbatim}
\endminipage
\minipage{4cm}
\begin{verbatim}
a, code {
  color: #f7a;
  font-size: 18px;
}
@media (max-width: 479px) {
  a, code {
    font-size: 14px;
  }
}
a:hover, code:hover {
  color: #ff92bb;
}
\end{verbatim}
\endminipage

\subsection{MongoDB}

\section{\eng{Front-end}}

\subsection{Backbone}

\subsection{jQuery}

\subsection{Ace}

\subsection{PDF.js}

\subsection{Bower}

\section{Comun \eng{backend} și \eng{frontend}}

\subsection{Share.JS}

\section{Autentificare}

Stocarea incorectă a parolelor încă este o problemă des întâlnită astăzi și a
dus la compromiterea a milioane de conturi pentru multe companii (vezi
tabelul~\ref{parole}).

Eu am folosit \eng{hashing} a parolei cu \eng{salt} (data înregistrării). O altă
variantă ar fi fost bcrypt, dar verificarea este mai computațional intensivă.

\begin{table}[hb]
\begin{center}
\begin{tabular}{l l r r}
Data & Compania & Parole unice & Parole totale \\
\hline
2012-07-11 & Yahoo & 342.508 & 442.832 \\
2012-02-22 & YP & 833.994 & 1.566.156 \\
2009-12-04 & RockYou & 14.344.173 & 32.603.145 \\
2011-12 & CSDN & 4.037.902 & 6.428.632 \\
\end{tabular}
\end{center}
\caption{Compromiterea parolelor stocate în clar\cite{passleak}.}
\label{parole}
\end{table}

\subsection{Sugestii}

Unele situri folosesc sau chiar obligă folosirea de sugestii pentru amintirea
parolelor. Eu nu am făcut asta deoarece în cazul compromiterii bazei de date
acest lucru ajută la găsirea parolelor. Acest lucru s-a întâmplat în 2013 când
hash-urile și sugestiile pentru 150 de milioane de conturi Adobe au fost
publicate. Deoarece nu se folosea \eng{salt}, dacă mai mulți utilizatori aveau
aceeași parolă, sugestiile ajută la ghicirea parolei.

\subsection{Părți terțe}

O variantă de autentificare este folosirea unor părți terțe. Eu nu am făcut asta
pentru că acest lucru implică dependența de alte sisteme care pot încetini
dezvoltarea rapidă.

\label{inregrap}
\subsection{Înregistrare rapidă}

În loc să permit autentificarea cu conturi de pe alte situri, am ales să permit
înregistrarea foarte rapidă. De pe prima pagina se poate realiza înregistrarea
scriind doar numele dorit și parola (și confirmarea ei).

\subsection{Probleme}

Deoarece nu am un certificat, încă nu folosesc HTTPS pe \cod{multilatex.com},
deci autentificarea nu este foarte sigură.

\section{Izolare \LaTeX}

\TeX este un limbaj puternic și deci posibil periculos. Trebuie avut grijă când
se execută cod arbitrar de la surse necunoscute.

Am scris un modul care să izoleze (\eng{sandboxing}) pe cât posibil compilarea
proiectelor \LaTeX.

\subsection{Execuția comenzilor de sistem}

Una dintre primele probleme este că \LaTeX{} poate să execute comenzi de sistem
arbitrare folosind construcția \cod{\textbackslash write18\{comandă\}}.

\idee{Arată un exemplu de comandă.}

In mod normal acestă construcție nu este disponibilă, dar pentru mai mare
siguranță poate fi oprită direct din execuția \cod{pdflatex} cu argumentul
\cod{-no-shell-escape}. \poate{Se mai folosește în plus argumentul
\cod{-halt-on-error} pentru a se opri execuția programului în loc să aștepte
corectarea interactivă a erorilor.}

\subsection{Cod rău-intenționat}

\TeX{} este un limbaj Turing-complet \idee{(dovadă)} deci nu se poate determina
fără a fi fie excutat codul dacă compilarea documentului se va termina.

\LaTeX{}, cu toate pachetele sale, este uriaș, deci foarte probabil există
probleme nedescoperite care pot duce la umplerea memoriei sau intrarea în bucle
infinite.

Problema terminării poate fi ameliorată prin folosirea unui \eng{timeout} pentru
compilarea documentelor, dar acest lucru poate fi problematic pentru documente
foarte mari.

În plus, se folosește un utilizator special pentru execuția programului
\cod{pdflatex} astfel încât să fie limitată compromiterea procesului prin codul
rulat.

Acest lucru permite în Linux și setarea de limite:

\begin{itemize}

\item spre unde poate scrie un utilizator;

\item câtă memorie virtuală poate utiliza (folosing \cod{ulimits});

\item cât timp de CPU poate folosi, și altele.

\end{itemize}

Codul rău nu este în mod necesar malițios deci un utilizator trebuie prezentat
cu o eroare în loc să fie interzis.

\subsection{Altele}

Mai sunt alte feluri în care se poate influența sistemul, spre exemplu prin
scrierea de fișiere folosind pachetul \cod{newfile}. Așa se poate umple sistemul
de fișiere. Și această problemă este rezolvată prin setarea de limite pe
utilizator.

\section{Compilator \LaTeX}

\section{\eng{File store}}

\section{Comunicare}

\section{Editor}

\label{sabloanesec}
\section{Șabloane}

\chapter{Rezultate}

\idee{JavaScript, + Browserify, + minificare}

\chapter{Îmbunătățiri}

\section{\eng{File store}}

\chapter{Aplicații}

\capfara{Concluzii}

\begin{thebibliography}{1}
\addcontentsline{toc}{chapter}{Bibliografie}

\bibitem{texweb}
\url{mirrors.ctan.org/systems/knuth/dist/tex/tex.web}

\bibitem{texwebpdf}
\url{brokestream.com/tex.pdf}

\bibitem{passleak}
\url{thepasswordproject.com/leaked\_password\_lists\_and\_dictionaries}

\bibitem{texlivejs}
\url{manuels.github.io/texlive.js}

\bibitem{statcount}
\url{gs.statcounter.com}

\end{thebibliography}

\end{document}

\documentclass[a4wide,12pt]{report}

\usepackage[romanian]{babel}
\usepackage[utf8x]{inputenc}
\usepackage{setspace}
\usepackage{graphicx}
\usepackage[top=3cm, bottom=3cm, left=3cm, right=3cm]{geometry}
\usepackage{tikz}
% Use Open Sans as the sans serif font.
\usepackage[defaultsans,osfigures,scale=0.95]{opensans}
\usepackage[T1]{fontenc}

\renewcommand{\seriesdefault}{l}

% Opțiuni la diagrame tikz.
\usetikzlibrary{patterns, trees, matrix, arrows, decorations.pathmorphing,
decorations.pathreplacing}

% Comenzi
\newcommand{\eng}[1]{\emph{#1}} % Pentru cuvinte în limba engleză.
\newcommand{\cod}[1]{\texttt{#1}}
\newcommand{\url}[1]{\texttt{#1}}
\newcommand{\uplu}[1]{$\langle$#1$\rangle$}
\newcommand{\floor}[1]{\lfloor{}#1\rfloor} % Pentru trunchere.
\newcommand{\acr}[1]{{\textsmaller[1]{\textsc{#1}}}} % Pentru acronime.


\newcommand{\capfara}[1]{
    \chapter*{#1}
    \addcontentsline{toc}{chapter}{#1}
}

\newcommand{\secfara}[1]{
    \section*{#1}
    \addcontentsline{toc}{section}{#1}
}

% Variabile
\newcommand{\subtitlu}{editor web colaborativ de \LaTeX}
\newcommand{\titlu}{Multilatex: \subtitlu}
\newcommand{\autor}{Paul Răzvan Nechifor} % Ăsta-s io. :)
\newcommand{\coordonator}{conf. dr. Sabin-Corneliu Buraga}
\newcommand{\sesiunea}{iulie, 2014}
\newcommand{\datacurenta}{99 iunie 2014}
\newcommand{\loculsisemnatura}{
    \vspace{20mm}
    \noindent
    Iași, \datacurenta
    \vspace{20mm}
    \begin{flushright}
        Absolvent \emph{\autor}\\
        \setlength{\unitlength}{1cm}
        \begin{picture}
            (4, 1)\put(0, 0)
            {\line(1, 0){4}}
        \end{picture}
        \vspace{5mm}\\
        (semnătura în original)
    \end{flushright}
}

\begin{document}
\frenchspacing % Don't use double spacing.
\onehalfspace % Larger line space.

\fontdimen2\font=5.00pt % Interword space (default 3.91663pt).
\fontdimen3\font=2.00pt % Interword stretch (default 1.95831pt).
\fontdimen4\font=1.50pt % Interword shrink (default 1.30554pt).
\fontdimen7\font=0.00pt % Extra space (default 1.30554pt).

\pagestyle{empty} % Hide page number.

% Cover background color.
\begin{tikzpicture}[overlay]
    \definecolor{albastru de fii}{rgb}{0.176, 0.447, 0.713}
    \fill [fill=albastru de fii] (-5,5) rectangle (21.0,-29.7);
\end{tikzpicture}

\begin{figure}[!htb]
\minipage{2cm}
    \includegraphics[width=2cm]{imagini/uaic}
\endminipage\hfill
\minipage{11cm}
    \centering
    \color{white}
    \large{\textsf{Universitatea Alexandru Ioan Cuza Iași}}

    \LARGE{\textsf{Facultatea de Informatică}}

    \large{\textsf{Master Ingineria Sistemelor Soft}}
\endminipage\hfill
\minipage{2cm}%
    % FCS logo.
    \begin{tikzpicture}[scale=0.26]
        \def\taie{0.5}
        \definecolor{alb}{rgb}{1, 1, 1}

        \fill [fill=alb]
            (0,0) -- (2,0) -- (2,3) -- (3,3) -- (5,5) -- (2,5) --
            (2,6) -- (6,6) -- (8,8) -- (0,8) -- cycle;
        \fill [fill=alb]
            (3,0) -- (3,3-\taie) -- (5,5-\taie) -- (5,0) -- cycle;
        \fill [fill=alb]
            (6,0) -- (6,6-\taie) -- (8,8-\taie) -- (8,0) -- cycle;
    \end{tikzpicture}
\endminipage
\end{figure}

\vspace{2cm}

\begin{center}
    \centering
    \color{white}
    {\fontsize{22}{26}\selectfont \textsf{LUCRARE DE DIZERTAȚIE}}\\
    \vspace{1.5cm}
    {\fontsize{40}{46}\selectfont \textsf{\textbf{Multilatex}}}\\
    \vspace{0.5cm}
    {\fontsize{26}{28}\selectfont \textsf{\subtitlu}}\\
    \vspace{1.5cm}
    \large{\textsf{propusă de}}\\
    \vspace{1.5cm}
    {\fontsize{26}{28}\selectfont \textsf{Paul Răzvan \textbf{Nechifor}}}\\
    \vspace{1.5cm}
    {\fontsize{16}{26}\selectfont \textsf{\textbf{Sesiunea:} \sesiunea}}\\
    \vspace{1.5cm}
    {\fontsize{14}{26}\selectfont \textsf{Coordonator științific}}\\
    {\fontsize{18}{26}\selectfont \textsf{\textbf{\coordonator}}}\\
\end{center}

\pagebreak

% Pagina de titlu
\begin{center}
    \textbf{UNIVERSITATEA ALEXANDRU IOAN CUZA IAȘI}\\
    \vspace{3mm}
    \textbf{FACULTATEA DE INFORMATICĂ}\\
    \vspace{89mm}
    \Huge{\textbf{\titlu}}\\
    \vspace{30mm}
    \Large{\textbf{\emph{\autor}}}\\
    \vspace{11mm}
    \Large{\textbf{Sesiunea:} \sesiunea}\\
    \vspace{10mm}
    \normalsize{Coordonator științific}\\
    \vspace{3mm}
    \large{\textbf{\coordonator}}\\
    \end{center}
\pagebreak

\section*{Declarație privind originalitate și respectarea drepturilor de autor}

Prin prezenta declar că lucrarea de licență cu titlul „\titlu“ este scrisă de
mine și nu a mai fost prezentată niciodată la o altă facultate sau instituție de
învățământ superior din țară sau străinătate. De asemenea, declar că toate
sursele utilizate, inclusiv cele preluate de pe internet, sunt indicate în
lucrare, cu respectarea regulilor de evitare a plagiatului:

\begin{itemize}
    \item toate fragmentele de text reproduse exact, chiar și în traducere
    proprie din altă limbă, sunt scrise între ghilimele și dețin referința
    precisă a sursei;
    \item reformularea în cuvinte proprii a textelor scrise de către alți autori
    deține referința precisă;
    \item codul sursă, imagini etc. preluate din proiecte cu sursă publică sau
    alte surse sunt utilizate cu respectarea drepturilor de autor și dețin
    referințe precise;
    \item rezumarea ideilor altor autori precizează referința precisă la textul
    original.
\end{itemize}

\loculsisemnatura

\pagebreak

\section*{Declarație de consimțământ}
Prin prezenta declar că sunt de acord ca lucrarea de licență cu titlul „\titlu“,
codul sursă al programelor și celelalte conținuturi (grafice, multimedia, date
de test etc.) care însoțesc această lucrare să fie utilizate în cadrul
Facultății de Informatică.

De asemenea, sunt de acord ca Facultatea de Informatică de la Universitatea
Alexandru Ioan Cuza Iași să utilizeze, modifice, reproducă și să distribuie în
scopuri necomerciale programele-calculator, format executabil și sursă,
realizate de mine în cadrul prezentei lucrări de licență.

\loculsisemnatura

\pagebreak

% Cuprins
\pagestyle{plain}
\setcounter{page}{1} % Începe numerotarea de la 1.

\tableofcontents

\pagebreak

\capfara{Introducere}

\secfara{Motivație}

\secfara{Challanges}

\capfara{Contribuții}

\chapter{Sisteme existente}

\chapter{Sistemul propriu}

\chapter{Rezultate}

\chapter{Îmbunătățiri}

\chapter{Aplicații}

\capfara{Concluzii}

\begin{thebibliography}{1}
\addcontentsline{toc}{chapter}{Bibliografie}

\end{thebibliography}

\end{document}

\documentclass[a4wide,12pt]{report}

\usepackage[romanian]{babel}
\usepackage[utf8x]{inputenc}
\usepackage{setspace}
\usepackage{graphicx}
\usepackage[top=3cm, bottom=3cm, left=3cm, right=3cm]{geometry}
\usepackage{tikz}
% Use Open Sans as the sans serif font.
\usepackage[defaultsans,osfigures,scale=0.95]{opensans}
\usepackage[T1]{fontenc}
\usepackage{color}

\renewcommand{\seriesdefault}{l}

% Opțiuni la diagrame tikz.
\usetikzlibrary{patterns, trees, matrix, arrows, decorations.pathmorphing,
decorations.pathreplacing}

% Comenzi
\newcommand{\eng}[1]{\emph{#1}} % Pentru cuvinte în limba engleză.
\newcommand{\cod}[1]{\texttt{#1}}
\newcommand{\url}[1]{\texttt{#1}}
\newcommand{\uplu}[1]{$\langle$#1$\rangle$}
\newcommand{\floor}[1]{\lfloor{}#1\rfloor} % Pentru trunchere.
\newcommand{\acr}[1]{{\textsmaller[1]{\textsc{#1}}}} % Pentru acronime.
\newcommand{\idee}[1]{{\color{red} #1}}
\newcommand{\poate}[1]{#1}

\newcommand{\capfara}[1]{
    \chapter*{#1}
    \addcontentsline{toc}{chapter}{#1}
}

\newcommand{\secfara}[1]{
    \section*{#1}
    \addcontentsline{toc}{section}{#1}
}

% Variabile
\newcommand{\subtitlu}{editor web colaborativ de \LaTeX}
\newcommand{\titlu}{Multilatex: \subtitlu}
\newcommand{\autor}{Paul Răzvan Nechifor} % Ăsta-s io. :)
\newcommand{\coordonator}{conf. dr. Sabin-Corneliu Buraga}
\newcommand{\sesiunea}{iulie, 2014}
\newcommand{\datacurenta}{99 iunie 2014}
\newcommand{\loculsisemnatura}{
    \vspace{20mm}
    \noindent
    Iași, \datacurenta
    \vspace{20mm}
    \begin{flushright}
        Absolvent \emph{\autor}\\
        \setlength{\unitlength}{1cm}
        \begin{picture}
            (4, 1)\put(0, 0)
            {\line(1, 0){4}}
        \end{picture}
        \vspace{5mm}\\
        (semnătura în original)
    \end{flushright}
}

\begin{document}
\frenchspacing % Don't use double spacing.
\onehalfspace % Larger line space.

\fontdimen2\font=5.00pt % Interword space (default 3.91663pt).
\fontdimen3\font=2.00pt % Interword stretch (default 1.95831pt).
\fontdimen4\font=1.50pt % Interword shrink (default 1.30554pt).
\fontdimen7\font=0.00pt % Extra space (default 1.30554pt).

\pagestyle{empty} % Hide page number.

% Cover background color.
\begin{tikzpicture}[overlay]
    \definecolor{albastru de fii}{rgb}{0.176, 0.447, 0.713}
    \fill [fill=albastru de fii] (-5,5) rectangle (21.0,-29.7);
\end{tikzpicture}

\begin{figure}[!htb]
\minipage{2cm}
    \includegraphics[width=2cm]{imagini/uaic}
\endminipage\hfill
\minipage{11cm}
    \centering
    \color{white}
    \large{\textsf{Universitatea Alexandru Ioan Cuza Iași}}

    \LARGE{\textsf{Facultatea de Informatică}}

    \large{\textsf{Master Ingineria Sistemelor Soft}}
\endminipage\hfill
\minipage{2cm}%
    % FCS logo.
    \begin{tikzpicture}[scale=0.26]
        \def\taie{0.5}
        \definecolor{alb}{rgb}{1, 1, 1}

        \fill [fill=alb]
            (0,0) -- (2,0) -- (2,3) -- (3,3) -- (5,5) -- (2,5) --
            (2,6) -- (6,6) -- (8,8) -- (0,8) -- cycle;
        \fill [fill=alb]
            (3,0) -- (3,3-\taie) -- (5,5-\taie) -- (5,0) -- cycle;
        \fill [fill=alb]
            (6,0) -- (6,6-\taie) -- (8,8-\taie) -- (8,0) -- cycle;
    \end{tikzpicture}
\endminipage
\end{figure}

\vspace{2cm}

\begin{center}
    \centering
    \color{white}
    {\fontsize{22}{26}\selectfont \textsf{LUCRARE DE DIZERTAȚIE}}\\
    \vspace{1.5cm}
    {\fontsize{40}{46}\selectfont \textsf{\textbf{Multilatex}}}\\
    \vspace{0.5cm}
    {\fontsize{26}{28}\selectfont \textsf{\subtitlu}}\\
    \vspace{1.5cm}
    \large{\textsf{propusă de}}\\
    \vspace{1.5cm}
    {\fontsize{26}{28}\selectfont \textsf{Paul Răzvan \textbf{Nechifor}}}\\
    \vspace{1.5cm}
    {\fontsize{16}{26}\selectfont \textsf{\textbf{Sesiunea:} \sesiunea}}\\
    \vspace{1.5cm}
    {\fontsize{14}{26}\selectfont \textsf{Coordonator științific}}\\
    {\fontsize{18}{26}\selectfont \textsf{\textbf{\coordonator}}}\\
\end{center}

\pagebreak

% Pagina de titlu
\begin{center}
    \textbf{UNIVERSITATEA ALEXANDRU IOAN CUZA IAȘI}\\
    \vspace{3mm}
    \textbf{FACULTATEA DE INFORMATICĂ}\\
    \vspace{89mm}
    \Huge{\textbf{\titlu}}\\
    \vspace{30mm}
    \Large{\textbf{\emph{\autor}}}\\
    \vspace{11mm}
    \Large{\textbf{Sesiunea:} \sesiunea}\\
    \vspace{10mm}
    \normalsize{Coordonator științific}\\
    \vspace{3mm}
    \large{\textbf{\coordonator}}\\
    \end{center}
\pagebreak

\section*{Declarație privind originalitate și respectarea drepturilor de autor}

Prin prezenta declar că lucrarea de licență cu titlul „\titlu“ este scrisă de
mine și nu a mai fost prezentată niciodată la o altă facultate sau instituție de
învățământ superior din țară sau străinătate. De asemenea, declar că toate
sursele utilizate, inclusiv cele preluate de pe internet, sunt indicate în
lucrare, cu respectarea regulilor de evitare a plagiatului:

\begin{itemize}
    \item toate fragmentele de text reproduse exact, chiar și în traducere
    proprie din altă limbă, sunt scrise între ghilimele și dețin referința
    precisă a sursei;
    \item reformularea în cuvinte proprii a textelor scrise de către alți autori
    deține referința precisă;
    \item codul sursă, imagini etc. preluate din proiecte cu sursă publică sau
    alte surse sunt utilizate cu respectarea drepturilor de autor și dețin
    referințe precise;
    \item rezumarea ideilor altor autori precizează referința precisă la textul
    original.
\end{itemize}

\loculsisemnatura

\pagebreak

\section*{Declarație de consimțământ}
Prin prezenta declar că sunt de acord ca lucrarea de licență cu titlul „\titlu“,
codul sursă al programelor și celelalte conținuturi (grafice, multimedia, date
de test etc.) care însoțesc această lucrare să fie utilizate în cadrul
Facultății de Informatică.

De asemenea, sunt de acord ca Facultatea de Informatică de la Universitatea
Alexandru Ioan Cuza Iași să utilizeze, modifice, reproducă și să distribuie în
scopuri necomerciale programele-calculator, format executabil și sursă,
realizate de mine în cadrul prezentei lucrări de licență.

\loculsisemnatura

\pagebreak

% Cuprins
\pagestyle{plain}
\setcounter{page}{1} % Începe numerotarea de la 1.

\tableofcontents

\pagebreak

\capfara{Introducere}

\secfara{Motivație}

\secfara{Challanges}

\capfara{Contribuții}

\chapter{Sisteme existente}

\section{\TeX}

\section{\LaTeX}

\section{ShareLatex}

\section{WriteLatex}

\chapter{Sistemul propriu}

\section{Proiectare}

\section{Structura sitului}

\section{\eng{Back-end}}

\subsection{Vagrant}

\subsection{Node}

\subsection{Express}

\subsection{Jade}

\subsection{Stylus}

Stylus (2010) este un limbaj \eng{stylesheet} care compilează în CSS. Este
asemănător cu Sass (2007) și LESS (2009), dar cu diferența principală că este
mult mai succint și este bazat pe indentare.

În următorul exemplu este arătat codul Stylus în stânga și rezultatul CSS în
dreapta. Se folosește o variabliă \cod{main-color} pentru culoare care este și
manipulată pentru \cod{:hover} (înalbită cu 20~\%). Folosind \cod{main-size}
mărimea textului pe dispozitive cu ecran mic este micșorată la 80~\%. Este
demonstrată și imbricarea care evită repetarea: se folosește doar \cod{\&:hover}
în loc de \cod{a:hover, code:hover}.

\minipage{9cm}
\begin{verbatim}
main-color = #f7a
main-size = 18px

a, code
  color main-color
  font-size main-size
  @media (max-width: 479px)
    font-size floor(main-size * 0.8)
  &:hover
    color lighten(main-color, 20%)
\end{verbatim}
\endminipage
\minipage{4cm}
\begin{verbatim}
a, code {
  color: #f7a;
  font-size: 18px;
}
@media (max-width: 479px) {
  a, code {
    font-size: 14px;
  }
}
a:hover, code:hover {
  color: #ff92bb;
}
\end{verbatim}
\endminipage

\subsection{MongoDB}

\subsection{Share.JS}

\section{\eng{Front-end}}

\subsection{Backbone}

\subsection{jQuery}

\subsection{Ace}

\subsection{PDF.js}

\section{Deployment}

\section{Autentificare}

Stocarea incorectă a parolelor încă este o problemă des întâlnită astăzi și a
dus la compromiterea a milioane de conturi pentru multe companii (vezi
tabelul~\ref{parole}).

Eu am folosit \eng{hashing} a parolei cu \eng{salt} (data înregistrării). O altă
variantă ar fi fost bcrypt, dar verificarea este mai computațional intensivă.

\begin{table}[hb]
\begin{center}
\begin{tabular}{l l r r}
Data & Compania & Parole unice & Parola totale \\
\hline
2012-07-11 & Yahoo & 342.508 & 442.832 \\
2012-02-22 & YP & 833.994 & 1.566.156 \\
2009-12-04 & RockYou & 14.344.173 & 32.603.145 \\
2011-12 & CSDN & 4.037.902 & 6.428.632 \\
\end{tabular}
\end{center}
\caption{Compromiterea parolelor stocate în clar\cite{passleak}.}
\label{parole}
\end{table}

\subsection{Sugestii}

Unele situri folosesc sau chiar obligă folosirea de sugestii pentru amintirea
parolelor. Eu nu am făcut asta deoarece în cazul compromiterii bazei de date
acest lucru ajută la găsirea parolelor. Acest lucru s-a întâmplat în 2013 când
hash-urile și sugestiile pentru 150 de milioane de conturi Adobe au fost
publicate. Deoarece nu se folosea \eng{salt}, dacă mai mulți utilizatori aveau
aceeași parolă, sugestiile ajută la ghicirea parolei.

\subsection{Părți terțe}

O variantă de autentificare este folosirea unor părți terțe. Eu nu am făcut asta
pentru că acest lucru implică dependența de alte sisteme care pot încetini
dezvoltarea rapidă.

\subsection{Înregistrare rapidă}

În loc să permit autentificarea cu conturi de pe alte situri, am ales să permit
înregistrarea foarte rapidă. De pe prima pagina se poate realiza înregistrarea
scriind doar numele dorit și parola (și confirmarea ei).

\subsection{Probleme}

Deoarece nu am un certificat, încă nu folosesc HTTPS pe \cod{multilatex.com},
deci autentificarea nu este foarte sigură.

\section{Izolare \LaTeX}

\TeX este un limbaj puternic și deci posibil periculos. Trebuie avut grijă când
se execută cod arbitrar de la surse necunoscute.

Am scris un modul care să izoleze (\eng{sandboxing}) pe cât posibil compilarea
proiectelor \LaTeX.

\subsection{Execuția comenzilor de sistem}

Una dintre primele probleme este că \LaTeX{} poate să execute comenzi de sistem
arbitrare folosind construcția \cod{\textbackslash write18\{comandă\}}.

\idee{Arată un exemplu de comandă.}

In mod normal acestă construcție nu este disponibilă, dar pentru mai mare
siguranță poate fi oprită direct din execuția \cod{pdflatex} cu argumentul
\cod{-no-shell-escape}. \poate{Se mai folosește în plus argumentul
\cod{-halt-on-error} pentru a se opri execuția programului în loc să aștepte
corectarea interactivă a erorilor.}

\subsection{Cod rău-intenționat}

\TeX{} este un limbaj Turing-complet \idee{(dovadă)} deci nu se poate determina
fără a fi fie excutat codul dacă compilarea documentului se va termina.

\LaTeX{}, cu toate pachetele sale, este uriaș, deci foarte probabil există
probleme nedescoperite care pot duce la umplerea memoriei sau intrarea în bucle
infinite.

Problema terminării poate fi ameliorată prin folosirea unui \eng{timeout} pentru
compilarea documentelor, dar acest lucru poate fi problematic pentru documente
foarte mari.

În plus, se folosește un utilizator special pentru execuția programului
\cod{pdflatex} astfel încât să fie limitată compromiterea procesului prin codul
rulat.

Acest lucru permite în Linux și setarea de limite:

\begin{itemize}
\item spre unde poate scrie un utilizator;
\item câtă memorie virtuală poate utiliza (folosing \cod{ulimits});
\item cât timp de CPU poate folosi, și altele.
\end{itemize}

Codul rău nu este în mod necesar malițios deci un utilizator trebuie prezentat
cu o eroare în loc să fie interzis.

\subsection{Altele}

Mai sunt alte feluri în care se poate influența sistemul, spre exemplu prin
scrierea de fișiere folosind pachetul \cod{newfile}. Așa se poate umple sistemul
de fișiere. Și această problemă este rezolvată prin setarea de limite pe
utilizator.

\section{File store}

\chapter{Rezultate}

\chapter{Îmbunătățiri}

\section{File store}

\chapter{Aplicații}

\capfara{Concluzii}

\begin{thebibliography}{1}
\addcontentsline{toc}{chapter}{Bibliografie}

\bibitem{passleak}
\url{http://thepasswordproject.com/leaked\_password\_lists\_and\_dictionaries}

\end{thebibliography}

\end{document}

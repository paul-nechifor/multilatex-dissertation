\documentclass{llncs}

\usepackage[utf8x]{inputenc}

\begin{document}

\title{Multilatex: a collaborative web editor for \LaTeX}

\author{Paul Nechifor}

\institute{Faculty of Computer Science,\\University of Iași}

\maketitle

\begin{abstract}

This report describes the building of a system for collaborative editing of \LaTeX{} projects. This includes simultaneous editing of multiple files, manipulating the history, ... Sharing features... This paper only describes the technical details and not all the features (which are reviewed in a demo on the website: \texttt{multilatex.com}).

\end{abstract}

\section{Introduction}

\TeX{} was started in 1977 by D. Knuth to better typeset scientific articles and books. \LaTeX{} is a generalised set of macros based on \TeX{}.

The advantages it has over WYSIWYG editors is that it allows you to focus on the content rather than its presentation. In those editors beginers are frequently tempted to insert spaces and new lines for alignment and page breaking because they are editing the presentation rather than the semantincs.

Although it as an ecelent tool at its inception in the limited systems of the day, \TeX{} fails to take advantace of more recent advancements. Some of the problems are:

\begin{itemize}

\item Static workflow. The usual workflow is: edit text file, compile, review output, repeat. \LaTeX{} doesn't compile what changed, it can only compile the whole document which is slow.

\item Package system. \LaTeX{} supports packages in a limited way. There is no versioning in requiring packages. Repositories are based on distributions.

\item ...

\end{itemize}

Some of these problems are solved by web sites like \mbox{ShareLaTeX} and \mbox{writeLaTeX}. In this paper I describe building a similar web site.

\section{Server}

\subsection{Stack}

The development and production websites run on Ubuntu 12.04, but other Linux distibutions can be used.

The server-side logic in written in JavaScript for Node.

Multilatex depends on the \texttt{texlive-full} package which requires about 2~GiB of disk space and provides a large selection of \LaTeX packages.

As a web framework I use Express which is the most widely used framework for Node.

I use Jade for HTML templates and Stylus as a stylesheet language. They have a similar clutter-free syntax based on whitespace.

\subsection {Build Process}

\section{Client}



\section{Conclusion}

Conclusions are here.

\begin{thebibliography}{1}

\bibitem{Einstein}
A. Einstein, On the movement of small particles suspended in stationary liquids required by the molecular-kinetic theory of heat, Annalen der Physik 17, pp. 549-560, 1905.

\end{thebibliography}

\end{document}

\documentclass[a4wide,12pt]{report}

\usepackage[romanian]{babel}
\usepackage{ucs}
\usepackage[utf8x]{inputenc}
\usepackage{setspace}
\usepackage{indentfirst}    % Ca să aliniez și primul paragraf.
\usepackage{fullpage}       % Ca să facă marginile mai mici.
\usepackage{relsize}        % Pentru mărimi relative de text.
\usepackage[top=3cm, bottom=3cm, left=3cm, right=3cm]{geometry}

% Comenzi
\newcommand{\eng}[1]{\emph{#1}} % Pentru cuvinte în limba engleză.
\newcommand{\cod}[1]{\texttt{#1}}
\newcommand{\acr}[1]{{\textsmaller[1]{\textsc{#1}}}} % Pentru acronime.

\begin{document}
\frenchspacing              % Ca să nu lase spațiu mai mare după punct.
\onehalfspace               % Ca să am spațiu mai mare între linii

\fontdimen2\font=5.00pt     % Interword space (default 3.91663pt).
\fontdimen3\font=2.00pt     % Interword stretch (default 1.95831pt).
\fontdimen4\font=1.50pt     % Interword shrink (default 1.30554pt).
\fontdimen7\font=0.00pt     % Extra space (default 1.30554pt).

\pagestyle{empty}

\begin{center}
    \mbox{}\\
    \vspace{1cm}
    {\Large{\textbf{Multilatex: editor web colaborativ de \LaTeX{} (rezumat)}}}\\
    \vspace{0.6cm}
    {\large{Paul Răzvan Nechifor}}\\
    \vspace{1cm}
\end{center}

Această lucrare prezintă proiectarea și construirea unei aplicații web de
administrare și editare a proiectelor \LaTeX{}.

Se începe cu introducerea rolului proiectului și motivația pentru construirea
lui.

În capitolul 1 se prezintă \LaTeX{} cu avantajele și dezavantajele sale și apoi
două sisteme similare în scop cu acest proiect.

În următorul capitol se introduc tehnologiile folosite pentru acest proiect și
motivația pentru folosirea lor.

Al treilea capitol este cel în care se prezintă sistemul propriu. Aici se începe
cu modul de proiectare a sistemului propus după care se prezintă structura
aplicației web. În acest capitol se mai vorbește și despre elementele de
integrare continuă, modul de izolare a codului \LaTeX{} și elemente particulare
ale aplicației cum ar fi sistemul de versionare propriu folosit și notificările
din proiecte. Spre final se prezintă funcționarea editorului propriu zis și a
modului de comunicare.

Urmează capitolul în care se sugerează modalități de reparare și îmbunătățire a
unor părți din sistemul construit.

În final, lucrarea se termină cu concluzii și lista bibliografică.

\vspace{0.6cm}
{\large{\textbf{}}}\\

\end{document}

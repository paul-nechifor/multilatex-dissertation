\documentclass[a4wide,12pt]{report}

\usepackage[romanian]{babel}
\usepackage{ucs}
\usepackage[utf8x]{inputenc}
\usepackage{setspace}
\usepackage{indentfirst}    % Ca să aliniez și primul paragraf.
\usepackage{fullpage}       % Ca să facă marginile mai mici.
\usepackage{relsize}        % Pentru mărimi relative de text.

% Comenzi
\newcommand{\eng}[1]{\emph{#1}} % Pentru cuvinte în limba engleză.
\newcommand{\cod}[1]{\texttt{#1}}
\newcommand{\acr}[1]{{\textsmaller[1]{\textsc{#1}}}} % Pentru acronime.

\begin{document}
\frenchspacing              % Ca să nu lase spațiu mai mare după punct.
\onehalfspace               % Ca să am spațiu mai mare între linii

\fontdimen2\font=5.00pt     % Interword space (default 3.91663pt).
\fontdimen3\font=2.00pt     % Interword stretch (default 1.95831pt).
\fontdimen4\font=1.50pt     % Interword shrink (default 1.30554pt).
\fontdimen7\font=0.00pt     % Extra space (default 1.30554pt).

\pagestyle{empty}

\begin{center}
    \mbox{}\\
    \vspace{1cm}
    {\Large{\textbf{Multilatex: editor web colaborativ de \LaTeX{} (rezumat)}}}\\
    \vspace{0.6cm}
    {\large{Paul Răzvan Nechifor}}\\
    \vspace{1cm}
\end{center}

Această lucrare prezintă proiectarea și construirea unei aplicații web de
administrare și editare a proiectelor \LaTeX{}.

Se începe cu

În capitolul 1 se prezintă \LaTeX{} cu avantajele și desavantajele sale și apoi
două sisteme existente similare în scop cu acest proiect.

În următorul capitol se prezintă

În capitolul trei se prezintă sistemul propriu.

Urmează capitolul de îmbunătățiri propuse

În final, lucrarea se termină cu concluzii și bibliografia.

\vspace{0.6cm}
{\large{\textbf{}}}\\

\end{document}
